\documentclass{hitec}
\usepackage{graphicx}
\usepackage{lscape}
\usepackage{longtable}
\usepackage{subcaption} 
\usepackage[space]{grffile}
\usepackage{pdfpages}
\usepackage{listings}
\usepackage{amsmath}
\definecolor{mygray}{rgb}{0.5,0.5,0.5}
\lstset{  breaklines=true, numbers=left, numberstyle=\tiny\color{mygray}, keepspaces=true }

%https://tex.stackexchange.com/questions/182467/including-eps-figure-in-pdflatex
\usepackage{epsfig}

\usepackage{siunitx} %https://tex.stackexchange.com/questions/413312/how-to-put-angstrom/413317

\usepackage{titlesec}
\usepackage{hyperref}
\usepackage{enumitem} %https://www.latex-tutorial.com/tutorials/lists/

%\usepackage{wasysym}
\usepackage{mathabx}

% https://shantoroy.com/latex/add-subfig-in-latex/
\usepackage{caption}
\usepackage{subcaption}

\usepackage{pgfplots} % https://www.tug.org/TUGboat/tb31-1/tb97wright-pgfplots.pdf

\usepackage{listings} % https://en.wikibooks.org/wiki/LaTeX/Source_Code_Listings

\usepackage[section]{placeins} % page 117 of Latex cookbook

% https://texblog.org/2012/05/30/generate-latex-tables-from-csv-files-excel/
%\usepackage{csvsimple}
%https://tex.stackexchange.com/questions/188556/importing-tabular-cells-from-text-file
%\newread\infile
%\def\preparetable#1#2{\bgroup \openin\infile=#1
%	\let\\=\relax \gdef\usetable{}\preparetableA #2,,}
%\def\preparetableA #1,{\if,#1,\egroup \closein\infile \else \read\infile to\tmp
%	\xdef\usetable{\usetable \tmp & #1 \\}\expandafter\preparetableA\fi}

\usepackage{natbib}

\usepackage{xcolor}
\hypersetup{
	colorlinks,
	linkcolor={blue!50!black},
	citecolor={blue!50!black},
	urlcolor={blue!80!black}
}

\usepackage{color, colortbl}
\definecolor{DSNE-Blue}{rgb}{0.773, 0.848, 0.941}
\definecolor{DSNE-Gray}{rgb}{0.848, 0.848, 0.848}

\titleclass{\subsubsubsection}{straight}[\subsection]

\newcounter{subsubsubsection}[subsubsection]
\renewcommand\thesubsubsubsection{\thesubsubsection.\arabic{subsubsubsection}}
\renewcommand\theparagraph{\thesubsubsubsection.\arabic{paragraph}} % optional; useful if paragraphs are to be numbered

\titleformat{\subsubsubsection}
{\normalfont\normalsize\bfseries}{\thesubsubsubsection}{1em}{}
\titlespacing*{\subsubsubsection}
{0pt}{3.25ex plus 1ex minus .2ex}{1.5ex plus .2ex}

\makeatletter
\renewcommand\paragraph{\@startsection{paragraph}{5}{\z@}%
	{3.25ex \@plus1ex \@minus.2ex}%
	{-1em}%
	{\normalfont\normalsize\bfseries}}
\renewcommand\subparagraph{\@startsection{subparagraph}{6}{\parindent}%
	{3.25ex \@plus1ex \@minus .2ex}%
	{-1em}%
	{\normalfont\normalsize\bfseries}}
\def\toclevel@subsubsubsection{4}
\def\toclevel@paragraph{5}
\def\toclevel@paragraph{6}
\def\l@subsubsubsection{\@dottedtocline{4}{7em}{4em}}
\def\l@paragraph{\@dottedtocline{5}{10em}{5em}}
\def\l@subparagraph{\@dottedtocline{6}{14em}{6em}}
\makeatother

\setcounter{secnumdepth}{4}
\setcounter{tocdepth}{4}


\title{Solar Cruiser Total Ionizing Dose}
\author{Anthony M. DeStefano}
\company{NASA, MSFC, EV44}
\confidential{\textbf{-- For internal NASA use only --}}
\usepackage{hyperref} 
\begin{document}
\maketitle
\pagenumbering{roman}

\tableofcontents
\listoffigures
\listoftables
%\lstlistoflistings
\newpage



%\section*{Contributing Author List}
%\addcontentsline{toc}{section}{Contributing Author List}



\cleardoublepage
\pagenumbering{arabic}
%%%%%%%%%%%%%%%%%%%%%%%%%%%%%%%%%%%%%%%%%%%%%%%%%%%%%%%%%%%%%%%%%%
%%%%%%%%%%%%%%%%%%%%%%%%%%%%%%%%%%%%%%%%%%%%%%%%%%%%%%%%%%%%%%%%%%
\section{Executive Summary}





\newpage
%%%%%%%%%%%%%%%%%%%%%%%%%%%%%%%%%%%%%%%%%%%%%%%%%%%%%%%%%%%%%%%%%%
%%%%%%%%%%%%%%%%%%%%%%%%%%%%%%%%%%%%%%%%%%%%%%%%%%%%%%%%%%%%%%%%%%
\section{Mission Trajectory}

The nominal Solar Cruiser location is beyond the Sun-Earth L1 point, called sub-L1, roughly $0.984$ AU from the Sun. The sub-L1 location is in interplanetary space with no shielding from any planetary bodies and is exposed to the solar wind. The nominal mission length is 2 years.

%%%%%%%%%%%%%%%%%%%%%%%%%%%%%%%%%%%%%%%%%%%%%%%%%%%%%%%%%%%%%%%%%%
%%%%%%%%%%%%%%%%%%%%%%%%%%%%%%%%%%%%%%%%%%%%%%%%%%%%%%%%%%%%%%%%%%
\section{Materials}

The representative material for the Reaction Control Device (RCD) of Solar Cruiser is $150 \mu\text{m}$ of Kapton. Using data from NIST\footnote{\url{https://physics.nist.gov/cgi-bin/Star/compos.pl?matno=179}}, the composition of Kapton is shown in Table \ref{tab:Kapton_composition}, where the density of Kapton is 1.42 g/cm$^3$.

\begin{table}[!h]\centering
	\caption{Composition of Kapton.}\label{tab:Kapton_composition}
	\begin{tabular}{|c | c | c | c | c |}\hline
		Atomic \# & Fraction by Weight & Weight (amu) & Fraction by Number & Stoichiometry\\\hline
		1	& 0.026362	&  1.008 & 0.25639 & 10\\\hline
		6	& 0.691133	& 12.011 & 0.56412 & 22\\\hline
		7	& 0.073270	& 14.007 & 0.05128 &  2\\\hline
		8	& 0.209235	& 15.999 & 0.12821 &  5\\\hline	
	\end{tabular}
\end{table}

%%%%%%%%%%%%%%%%%%%%%%%%%%%%%%%%%%%%%%%%%%%%%%%%%%%%%%%%%%%%%%%%%%
%%%%%%%%%%%%%%%%%%%%%%%%%%%%%%%%%%%%%%%%%%%%%%%%%%%%%%%%%%%%%%%%%%
\section{Dose-Depth in Kapton}

The SRIM (Stopping and Range of Ions in Solids) software is used to compute the dose in thin materials with no prior shielding. The SRIM software package contains TRIM (the Transport of Ions in Matter), with a screenshot of the setup used in this analysis shown in Figure \ref{fig:SRIM_setup}.

\begin{figure}[htbp!]
	\centering
	\includegraphics[width=1\textwidth]{../SRIM_setup.PNG}
	\caption{Example of TRIM setup.}\label{fig:SRIM_setup}
\end{figure}

The dose deposited in Kapton depends on the energy of the incident protons. As the energy increases, the depth of the deposited dose increases. In Figure \ref{fig:dose_depth_percentile_vs_energy}, the colored solid curves show the percentiles of dose deposited less than a particular depth, as a function of energy for normally incident protons. For example, the purple curve shows the depth vs. energy at which $95\%$ of the dose is deposited. From Figure~\ref{fig:dose_depth_percentile_vs_energy}, it is clear that the dose is not deposited  uniformly. Thicknesses between the green and blue curves, $45\%$ of the dose is deposited, as well as between the green and purple curves. Therefore, $90\%$ of the dose is deposited between the blue and purple curves. In general, the depth vs. energy profile has a double-power-law shape
\begin{equation}\label{eq:double-power-law}
	D(E) = \left(\frac{E}{a}\right)^b + \left(\frac{E}{c}\right)^d,
\end{equation}
where $a$ is the low-energy scale, $c$ is the high-energy scale, and $b$ \& $d$ are the indices for each scale, respectively. As the percentile increases, the values for the energy scales $a$ and $c$ decreases. For a list of parameter fits, see Table \ref{tab:double-power-law-parameters} in Appendix \ref{App:Dose Percentile-Depth vs. Energy Parameters Fits}. Observing that $b < d$, the Kapton material is more effective at stopping lower energy protons ($ < \sim 100$ keV) than higher energy protons.

\begin{figure}[htbp!]
	\centering
	\includegraphics[width=0.95\textwidth]{../Bragg_peak_data/dose_depth_percentile_vs_energy.png}
	\caption{Left axis: Depth in Kapton vs.\ energy for various dose-depth percentiles. Right axis: Dose per fluence vs.\ energy.}\label{fig:dose_depth_percentile_vs_energy}
\end{figure}


The dotted-dashed curve in Figure \ref{fig:dose_depth_percentile_vs_energy} shows the total dose deposited per fluence as a function of energy. Basically, this curve can be thought of as the dose cross-section of Kapton for normally incident protons. One way to find the total dose for a given energy spectrum, one could convolute the dose cross-section with the differential energy spectrum, taking into account the depth percentile for a particular thickness of Kapton material. However, a more direct method to find the total dose is used in the following sections.

%%%%%%%%%%%%%%%%%%%%%%%%%%%%%%%%%%%%%%%%%%%%%%%%%%%%%%%%%%%%%%%%%%
%%%%%%%%%%%%%%%%%%%%%%%%%%%%%%%%%%%%%%%%%%%%%%%%%%%%%%%%%%%%%%%%%%
\section{Natural Environment}

The natural environments for the RCD of Solar Cruiser are separated into a mid-energy component (solar energetic particles, Section~\ref{ssec:natenv-Solar Particle Events}) and a low-energy component (background solar wind, Section~\ref{ssec:natenv-Low-Energy Solar Wind}). The high-energy component (galactic cosmic rays, $> \text{GeV}$) are omitted because of the thin materials studied in this analysis. In general, for an isotropic proton environment with energies greater than $\sim 6.5$~MeV (see Equation~\eqref{eq:double-power-law}), the particles do not deposit a significant amount of energy in $150 \mu$m Kapton (i.e., the Bragg peak has not been reached yet).


%%%%%%%%%%%%%%%%%%%%%%%%%%%%%%%%%%%%%%%%%%%%%%%%%%%%%%%%%%%%%%%%%%
%%%%%%%%%%%%%%%%%%%%%%%%%%%%%%%%%%%%%%%%%%%%%%%%%%%%%%%%%%%%%%%%%%
\subsection{Solar Particle Events}
\label{ssec:natenv-Solar Particle Events}

The solar particle event (SPE) environment for interplanetary space is derived following the same procedure as outlined in the Cross-Program Design Specification for Natural Environments (DSNE) Section 3.3.1 (see the technical notes at the end of the section). A 2-year trajectory is defined in interplanetary space at the sub-L1 location of 0.984 AU in SPENVIS\footnote{\url{https://www.spenvis.oma.be/}} under the \textsf{Coordinate generators} tab. Once this is set, the SPE fluence is computed under the \textsf{Solar particle mission fluences}. The following parameters are set:
\begin{itemize}
	\item Solar particle model: ESP-PSYCHIC (worst event fluence)
	\item Ion range: H to H
	\item Prediction period: override
	\item Prediction period [years]: 2.0
	\item Offset in solar cycle: override
	\item Offset from solar maximum [years]: 0
	\item Confidence level [\%]: 95.0
\end{itemize}

Table \ref{tab:ESP-PSYCHIC_2-year_subL1} shows the down-selected energy bins that match DSNE Table 3.3.1.10.2-1. In terms of fluence $> 100$ keV, the 2-year SPE environment shown in Table \ref{tab:ESP-PSYCHIC_2-year_subL1} is only $6\%$ greater than the unshielded SPE environment is DSNE Table 3.3.1.10.2-1, despite the fluence duration doubling. This is why it is important to run the ESP-PSYCHIC model for the required mission length and not multiply Table 3.3.1.10.2-1 by the number of years. Otherwise, a large overestimate occurs. 

\begin{table}[!h]\centering
	\caption{ESP-PSYCHIC worst event fluence of protons for 2 years during solar maximum at 0.984 AU. The energy center $=\sqrt{\text{bin left edge}\times\text{bin right edge}.}$ }\label{tab:ESP-PSYCHIC_2-year_subL1}
	\begin{tabular}{|c | c | c | c |}\hline
		Energy Center (keV) & Bin Flux (\#/cm$^2$) & Bin Width (keV) & Bin Left Edge (keV) \\\hline

1.58E+02&2.08E+11&1.50E+02&1.00E+02\\\hline
3.54E+02&1.00E+11&2.50E+02&2.50E+02\\\hline
7.07E+02&6.83E+10&5.00E+02&5.00E+02\\\hline
1.41E+03&4.65E+10&1.00E+03&1.00E+03\\\hline
2.65E+03&2.74E+10&1.50E+03&2.00E+03\\\hline
4.18E+03&1.46E+10&1.50E+03&3.50E+03\\\hline
5.96E+03&1.51E+10&2.10E+03&5.00E+03\\\hline
7.54E+03&4.83E+09&9.00E+02&7.10E+03\\\hline
8.49E+03&4.23E+09&1.00E+03&8.00E+03\\\hline
9.49E+03&3.38E+09&1.00E+03&9.00E+03\\\hline
1.26E+04&1.07E+10&6.00E+03&1.00E+04\\\hline
1.70E+04&1.88E+09&2.00E+03&1.60E+04\\\hline
1.90E+04&1.52E+09&2.00E+03&1.80E+04\\\hline
2.24E+04&3.50E+09&5.00E+03&2.00E+04\\\hline
2.96E+04&3.77E+09&1.00E+04&2.50E+04\\\hline
3.74E+04&1.83E+09&5.00E+03&3.50E+04\\\hline
4.24E+04&1.02E+09&5.00E+03&4.00E+04\\\hline
4.74E+04&1.15E+09&5.00E+03&4.50E+04\\\hline
5.96E+04&2.35E+09&2.10E+04&5.00E+04\\\hline
7.54E+04&5.35E+08&9.00E+03&7.10E+04\\\hline
8.49E+04&4.18E+08&1.00E+04&8.00E+04\\\hline
9.49E+04&2.59E+08&1.00E+04&9.00E+04\\\hline
1.26E+05&6.29E+08&6.00E+04&1.00E+05\\\hline
1.70E+05&7.92E+07&2.00E+04&1.60E+05\\\hline
1.90E+05&5.44E+07&2.00E+04&1.80E+05\\\hline
2.24E+05&7.92E+07&5.00E+04&2.00E+05\\\hline
3.16E+05&7.33E+07&1.50E+05&2.50E+05\\\hline
4.47E+05&1.27E+07&1.00E+05&4.00E+05\\\hline
	\end{tabular}
\end{table}
\clearpage % forces the figure to drop before this point

%%%%%%%%%%%%%%%%%%%%%%%%%%%%%%%%%%%%%%%%%%%%%%%%%%%%%%%%%%%%%%%%%%
%%%%%%%%%%%%%%%%%%%%%%%%%%%%%%%%%%%%%%%%%%%%%%%%%%%%%%%%%%%%%%%%%%
\subsection{Low-Energy Solar Wind}
\label{ssec:natenv-Low-Energy Solar Wind}

To compute the low-energy solar wind plasma contribution of the environment (assumed at 1 AU), the L2-CPE V1.3 software package was used. The proton fluence was computed with the setup shown in Figure \ref{fig:L2-CPE-Setup}. Other percentiles that are automatically calculated are $5\%$, $50\%$, $95\%$, and the maximum, mean, and minimum fluxes for each energy bin (see Listing \ref{lst:FLX_PTNXPOS_SWMX_IMP} in Appendix \ref{App:Raw L2-CPE output of the Interplanetary Proton Environment}). Table \ref{tab:p95_sunward_solarwind_IMP} shows the reduced data in the same format as Table \ref{tab:ESP-PSYCHIC_2-year_subL1}. The $95\%$ is used in accordance with DSNE (see the technical notes at the end of Section 3.3.1). The sunward facing flux is used and assumed isotropically as worst-case. 

\begin{figure}[htbp!]
	\centering
	\includegraphics[width=0.95\textwidth]{../L2CPE/L2-CPE-Setup.PNG}
	\caption{Screen shot of the setup used in computing the low-energy proton flux environment in L2-CPE.}\label{fig:L2-CPE-Setup}
\end{figure}




\begin{table}[!h]\centering
	\caption{The $95\%$ sunward solar wind environment from L2-CPE for 2 years.}\label{tab:p95_sunward_solarwind_IMP}
	\begin{tabular}{|c | c | c | c |}\hline
		Energy Center (keV) & Bin Flux (\#/cm$^2$) & Bin Width (keV) & Bin Left Edge (keV) \\\hline
1.34E-03&4.97E+09&8.00E-04&1.00E-03\\\hline
2.40E-03&8.36E+09&1.40E-03&1.80E-03\\\hline
4.23E-03&1.38E+10&2.40E-03&3.20E-03\\\hline
7.48E-03&2.24E+10&4.40E-03&5.60E-03\\\hline
1.33E-02&3.50E+10&7.80E-03&1.00E-02\\\hline
2.37E-02&5.29E+10&1.38E-02&1.78E-02\\\hline
4.21E-02&7.66E+10&2.46E-02&3.16E-02\\\hline
7.50E-02&1.05E+11&4.38E-02&5.62E-02\\\hline
1.33E-01&1.37E+11&7.78E-02&1.00E-01\\\hline
2.37E-01&1.64E+11&1.38E-01&1.78E-01\\\hline
4.22E-01&1.86E+11&2.46E-01&3.16E-01\\\hline
7.50E-01&1.94E+11&4.38E-01&5.62E-01\\\hline
1.33E+00&1.88E+11&7.78E-01&1.00E+00\\\hline
2.37E+00&1.68E+11&1.38E+00&1.78E+00\\\hline
4.22E+00&1.40E+11&2.46E+00&3.16E+00\\\hline
7.50E+00&1.10E+11&4.38E+00&5.62E+00\\\hline
1.33E+01&8.21E+10&7.78E+00&1.00E+01\\\hline
2.37E+01&5.75E+10&1.38E+01&1.78E+01\\\hline
4.22E+01&3.91E+10&2.46E+01&3.16E+01\\\hline
7.50E+01&2.56E+10&4.38E+01&5.62E+01\\\hline
1.33E+02&1.65E+10&7.78E+01&1.00E+02\\\hline
2.37E+02&1.04E+10&1.38E+02&1.78E+02\\\hline
4.22E+02&6.43E+09&2.46E+02&3.16E+02\\\hline
7.50E+02&3.91E+09&4.38E+02&5.62E+02\\\hline
1.33E+03&2.35E+09&7.78E+02&1.00E+03\\\hline
2.37E+03&1.43E+09&1.38E+03&1.78E+03\\\hline
4.22E+03&8.50E+08&2.46E+03&3.16E+03\\\hline
7.50E+03&5.05E+08&4.38E+03&5.62E+03\\\hline
	\end{tabular}
\end{table}
\clearpage % forces the figure to drop before this point

%%%%%%%%%%%%%%%%%%%%%%%%%%%%%%%%%%%%%%%%%%%%%%%%%%%%%%%%%%%%%%%%%%
%%%%%%%%%%%%%%%%%%%%%%%%%%%%%%%%%%%%%%%%%%%%%%%%%%%%%%%%%%%%%%%%%%
\section{Total Ionizing Dose}

%%%%%%%%%%%%%%%%%%%%%%%%%%%%%%%%%%%%%%%%%%%%%%%%%%%%%%%%%%%%%%%%%%
%%%%%%%%%%%%%%%%%%%%%%%%%%%%%%%%%%%%%%%%%%%%%%%%%%%%%%%%%%%%%%%%%%
\subsection{Solar Particle Events}


%%%%%%%%%%%%%%%%%%%%%%%%%%%%%%%%%%%%%%%%%%%%%%%%%%%%%%%%%%%%%%%%%%
%%%%%%%%%%%%%%%%%%%%%%%%%%%%%%%%%%%%%%%%%%%%%%%%%%%%%%%%%%%%%%%%%%
\subsection{Low-Energy Solar Wind}

%%%%%%%%%%%%%%%%%%%%%%%%%%%%%%%%%%%%%%%%%%%%%%%%%%%%%%%%%%%%%%%%%%
%%%%%%%%%%%%%%%%%%%%%%%%%%%%%%%%%%%%%%%%%%%%%%%%%%%%%%%%%%%%%%%%%%
\section{Results}

\appendix

\section{Dose Percentile-Depth vs. Energy Parameters Fits}
\label{App:Dose Percentile-Depth vs. Energy Parameters Fits}

\begin{table}[!h]\centering
	\caption{Parameter fits to Equation \eqref{eq:double-power-law} for various dose percentiles using normally incident protons.}\label{tab:double-power-law-parameters}
	\begin{tabular}{|c | c | c | c | c |}\hline
		Percentile/100 & $a$ & $b$ & $c$ & $d$ \\\hline

0.05&8.445E+00&6.539E-01&4.543E+03&1.785E+00\\\hline
0.10&4.414E+00&6.654E-01&3.267E+03&1.786E+00\\\hline
0.15&2.923E+00&6.729E-01&2.682E+03&1.787E+00\\\hline
0.20&2.171E+00&6.793E-01&2.339E+03&1.788E+00\\\hline
0.25&1.723E+00&6.845E-01&2.109E+03&1.789E+00\\\hline
0.30&1.429E+00&6.894E-01&1.947E+03&1.790E+00\\\hline
0.35&1.216E+00&6.937E-01&1.826E+03&1.792E+00\\\hline
0.40&1.057E+00&6.976E-01&1.732E+03&1.793E+00\\\hline
0.45&9.332E-01&7.013E-01&1.658E+03&1.794E+00\\\hline
0.50&8.296E-01&7.043E-01&1.598E+03&1.795E+00\\\hline
0.55&7.407E-01&7.067E-01&1.549E+03&1.796E+00\\\hline
0.60&6.660E-01&7.089E-01&1.510E+03&1.798E+00\\\hline
0.65&6.006E-01&7.109E-01&1.479E+03&1.799E+00\\\hline
0.70&5.423E-01&7.128E-01&1.456E+03&1.800E+00\\\hline
0.75&4.868E-01&7.140E-01&1.438E+03&1.801E+00\\\hline
0.80&4.309E-01&7.141E-01&1.423E+03&1.802E+00\\\hline
0.85&3.744E-01&7.131E-01&1.412E+03&1.803E+00\\\hline
0.90&3.140E-01&7.104E-01&1.402E+03&1.804E+00\\\hline
0.95&2.441E-01&7.046E-01&1.396E+03&1.804E+00\\\hline
0.99&1.554E-01&6.904E-01&1.393E+03&1.805E+00\\\hline
0.999&9.998E-02&6.752E-01&1.394E+03&1.807E+00\\\hline
0.9999&7.282E-02&6.638E-01&1.385E+03&1.806E+00\\\hline

	\end{tabular}
\end{table}

\section{Raw L2-CPE output of the Interplanetary Proton Environment}
\label{App:Raw L2-CPE output of the Interplanetary Proton Environment}

\lstinputlisting[language=xml,basicstyle=\footnotesize,frame=single,
caption={The sunward facing flux (worst-case) during solar maximum from IMP8 using L2-CPE.},label=lst:FLX_PTNXPOS_SWMX_IMP]{../L2CPE/FLX_PTNXPOS_SWMX_IMP.DAT}

%\section{References}
%\cleardoublepage
%\phantomsection
%\addcontentsline{toc}{section}{References}
%\bibliographystyle{agu}
%\bibliography{report}

\end{document}